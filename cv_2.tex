%%%%%%%%%%%%%%%%%%%%%%%%%%%%%%%%%%%%%%%%%
% Medium Length Professional CV
% LaTeX Template
% Version 2.0 (8/5/13)
%
% This template has been downloaded from:
% http://www.LaTeXTemplates.com
%
% Original author:
% Trey Hunner (http://www.treyhunner.com/)
%
% Important note:
% This template requires the resume.cls file to be in the same directory as the
% .tex file. The resume.cls file provides the resume style used for structuring the
% document.
%
%%%%%%%%%%%%%%%%%%%%%%%%%%%%%%%%%%%%%%%%%

%----------------------------------------------------------------------------------------
%	PACKAGES AND OTHER DOCUMENT CONFIGURATIONS
%----------------------------------------------------------------------------------------

\documentclass{resume} % Use the custom resume.cls style

\usepackage[left=0.75in,top=1.70in,right=0.75in,bottom=0.6in]{geometry} % Document margins

\name{Syamantak Naskar} % Your name
\address{64, Debinibash Road \\ Flat - 'O', 4th. Floor \\ Kolkata - 700 074} % Your address
\address{ (+91) 91674 69824 \\ syamantaknaskar@gmail.com \\ https://github.com/syamantaknaskar}
\begin{document}

%----------------------------------------------------------------------------------------
%	EDUCATION SECTION
%----------------------------------------------------------------------------------------

\begin{rSection}{Education}
\begin{rSubsection}{}{}{}{}
\item B.Tech in Computer Science and Engineering at IIT Bombay with GPA 6.27 from 2012-2016
\item Secured 91.2\% in Intermediate/+2 in CBSE Board in 2012
\item Secured 95.8\% in Matriculation in ICSE Board in 2010 
\end{rSubsection}
\end{rSection}


%----------------------------------------------------------------------------------------
%	SCHOLASTIC ACHIEVEMENTS SECTION
%----------------------------------------------------------------------------------------

\begin{rSection}{Scholastic Achievements}
\begin{rSubsection}{}{}{}{}
\item Secured All India Rank 2654 out of over half a million candidates in the IIT-JEE 2012
\item Secured All India Rank 3764 out of over 1.3 million candidates in the AIEEE 2012
\item Awarded scholarship for All India Talent Search Examination by International Children Equity Fund in 2007
\end{rSubsection}
\end{rSection}

%----------------------------------------------------------------------------------------
%	WORK EXPERIENCE SECTION
%----------------------------------------------------------------------------------------

\begin{rSection}{Internships}
\begin{rSubsection}{Quad Hashing}{Summer 2015}{Edelweiss Financial Services Limited}{Ajit Borate}
\item Programming Language Used : C++
\item Implemented a Quad Hash and measured time in nanoseconds
\item It performed better than Google Dense Map and Unordered Map in the STL Library
\end{rSubsection}
\end{rSection}

%----------------------------------------------------------------------------------------
%	COURSE PROJECTS
%----------------------------------------------------------------------------------------

\begin{rSection}{PROJECTS}

%------------------------------------------------

\begin{rSubsection}{Sales Prediction in E-Commerce Sites}{Spring 2015}{Machine Learning }{Prof. Ganesh Ramakrishnan, IIT Bombay}
\item Programming Language Used : Python, GNUPlot
\item Predicted the amount of sales of a product on given day and in given time interval using the previous sales data
for that particular product
\item Used Support Vector Regression (SVR) to model and train our data and analyse at various data points
\item Plotted graphs to identify the trends in consumer behaviour so that we can manage the stock accordingly
\end{rSubsection}
%------------------------------------------------


\begin{rSubsection}{Auto Parking Bot}{Spring 2015}{Embedded Systems Lab }{ Prof. Kavi Arya, IIT Bombay}
\item Programming Language Used : C++, Python
\item Made a car parking bot using TIVA Board which would recognise the type of vehicle and assign a parking slot to it if available
\item Used image processing to detect the type of vehicle and assign that particular type of spot to it
\item Kept track of time the vehicle is parked and charge the owner accordingly
\end{rSubsection}

%------------------------------------------------

\begin{rSubsection}{Virtual Memory Implementation}{Spring 2015}{Operating Systems }{Prof. D. M. Dhamdhere, IIT Bombay}
\item Programming Language Used : C++
\item Added virtual memory manager to GeekOS, an experimental OS built on top of Linux
\item Implemented modules for handling processes, memory allocation, swap space management, page replacement strategy (one handed clock)
\end{rSubsection}
%------------------------------------------------

\begin{rSubsection}{Compiler}{Spring 2014}{ Compilers}{Prof. Amitabha Sanyal, IIT Bombay}
\item Programming Language Used : BisonC++,  FlexC++ , C++
\item Took syntactically correct C++ program and converted it into optimal machine code using the least number of registers
\item Used Sethi-Ullman Algorithm to generate code for expression trees
\end{rSubsection}
%------------------------------------------------

\begin{rSubsection}{Performance Analysis Of B+ Tree}{Autumn 2014}{Database and Information Systems }{ Prof. N. L. Sarda, IIT Bombay}
\item Programming Language Used : C
\item Used Bulk Loading to load the entries in the B+ tree
\item Analysed the advantages of Concurrency control, fewer I/O to build and control fill factor in pages
\end{rSubsection}
%------------------

\begin{rSubsection}{Railway Timetable}{Autumn 2014}{Database and Information Systems }{ Prof. N. L. Sarda, IIT Bombay}
\item Programming Language Used : Java, PgSQL
\item Made a ER Diagram and Efficient DB design
\item Built a user friendly website to book tickets and cancel tickets
\item Simulated tracking of trains and ticket reservation system
\end{rSubsection}

%------------------------------------------------

\begin{rSubsection}{Web.DO}{Summer 2014}{ Windows Store App }{Microsoft Mentors}
\item Programming Language Used : Javascript(JS)
\item Created a Windows App which was a WYSIWYG website builder
\end{rSubsection}

%------------------------------------------------

\begin{rSubsection}{Machine Simulation using Box2D}{Spring 2013}{ Software Systems }{ Prof. Parag Chaudhuri, IIT Bombay}
\item Programming Language Used : Box2D, C++, GNUPlot
\item Simulated working of a piston using 2D physics engine Box2D with keyboard controls 
\item Optimized the code by profiling and identifying the time consuming parts

\end{rSubsection}

%------------------------------------------------

\begin{rSubsection}{Surveillance Bot}{Summer 2013}{ Institute Summer Technical Project }{ ITSP Mentors}
\item Programming Language Used : C++
\item Implemented Image Processing using OpenCV in Codeblocks IDE
\item Captures Captures images and records video in case of motion
\end{rSubsection}

%------------------------------------------------

\begin{rSubsection}{Backgammon Game}{Spring 2013}{ Abstraction and Paradigms for Programming }{ Prof, Amitabha Sanyal, IIT Bombay}
\item Programming Language Used : Scheme(Lisp)
\item Used the graphics and GUI libraries
\item Implemented Artificial Intelligence (AI) using minimax algorithm
\end{rSubsection}

%---------------------------------------------------

\begin{rSubsection}{Copter}{Autumn 2012}{Computer Programming and Utilization }{ Prof. Abhiram Ranade, IIT Bombay}
\item Programming Language Used : C++
\item Created a single player game in C++ with GUI in which you have to traverse through obstacles
\item Implemented GUI using EzWindows
\end{rSubsection}

\end{rSection}

%----------------------------------------------------------------------------------------
%	Position of Responsibility
%----------------------------------------------------------------------------------------

\begin{rSection}{Position of Responsibility}

%----------------------------------------------
\begin{rSubsection}{Convenor}{April 2013-14}{Web and Coding Club }{Students' Technical Activites Body, IIT Bombay}
\item Organising all web and coding events in IIT Bombay
\item Actively working in creation of Web and Coding Club website
\end{rSubsection}
%------------------------------------------------


%----------------------------------------------
\begin{rSubsection}{Organisor}{April 2012-13}{Publicity Department}{ Techfest 2013, IIT Bombay}
\item Coordinating with the colleges across the country for organising events
\item Organising and dealing with sponsors for the successful festival
\end{rSubsection}
%------------------------------------------------

\end{rSection}

%----------------------------------------------------------------------------------------
%	Technical Skills
%----------------------------------------------------------------------------------------

\begin{rSection}{Technical Skills}

%----------------------------------------------
\begin{rSubsection}{}{}{}{}
\item Skilled in C++ and object oriented concepts
\item Skilled in Scheme (Lisp) and concepts of Functional Programming
\item Familiar with OpenCV, MIPS, HTML, CSS, Scilab, Prolog, Python, C, Bash and Java
\item Platforms : Linux, Windows and Android
\end{rSubsection}
%------------------------------------------------

\end{rSection}

%----------------------------------------------------------------------------------------
%	MISCELLANEOUS
%----------------------------------------------------------------------------------------

\begin{rSection}{Miscellaneous}
\begin{rSubsection}{}{}{}{}
\item Built a wireless remote controlled bot for XLR8, a technical event in IITB
\item Have avid interest in competitive programming
\item Awarded the 3rd Kyu (Brown Belt) in Karate by Shito Ryu Karate-Do Federation
\item Worked under Green Campus as a part of NSS team planting trees in the Institute
\end{rSubsection}
\end{rSection}

%----------------------------------------------------------------------------------------
%	EXAMPLE SECTION
%----------------------------------------------------------------------------------------

%\begin{rSection}{Section Name}

%Section content\ldots

%\end{rSection}

%----------------------------------------------------------------------------------------

\end{document}

